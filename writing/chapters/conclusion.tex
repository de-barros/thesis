%!TEX root = ../thesis.tex
\chapter{Conclusion}
\label{conclusion}

The Discrete Voter Model (DVM) is a novel statistical method for performing ecological inference. The center of this method is the probabilistic hypercube (PHC), a rich data structure that encodes complex voting patterns in elections and electorates.

DVM leverages the power of discretization and Markov chain Monte Carlo to provide more expressive models within a more malleable and intuitive structure. In lower dimensions, the results of DVM can be easily visualized to provide more information about the distribution of possibilities of individual-level behavior than point estimates ever could. Even in higher dimensions, where visualization is not possible, the results of DVM are still quite expressive, and the distribution can be used to discover more about the data.

While DVM was developed for use on elections, it can be thought of and applied more foundationally as a tool for all kinds of ecological inference. If one would like to infer individual-level behavior when only aggregate group data is available, DVM can do that.

DVM is also built on state-of-the-art computing paradigms, like the dataflow and probabilistic programming provided by TensorFlow Probability. This ensures that the model, and the algorithms it implements, continues to leverage novel technologies and improvements in computing architecture.

The development of DVM was inspired by the pressing need to address the chronic problem of racial gerrymandering. Everyone, regardless of race or other personal identifiers, ought to be able to exercise their right to vote, and have that vote count towards change. A country will never be a true democracy until all barriers to voting, including more insidious and systemic ones like a gerrymandered electoral map, are destroyed. Hopefully, by helping to identify egregious racially polarized voting across the country, the Discrete Voter Model contributes positively to that effort.
